\documentclass{article}
\usepackage[left=2.5cm, right=2.5cm, top=3cm, bottom=3cm]{geometry}
\usepackage[utf8]{inputenc}
\usepackage{amssymb}
\usepackage{amsmath}

\title{Champernowne Constant\[ C_{10}\]}
\begin{document}
\maketitle
\section*{Brief introduction about the Interviewee}
{\bf Name} : Megha Kamath \\
{\bf Qualification} : Masters in Mathematics \\
{\bf Reason behind opting the interviewee} : My Interviewee is pursuing Masters in Mathematics and it was obvious for me to work with her to better understand about the Champernowne constant and its applications.
\section*{Interview Questions on Champernowne Constant}
\begin{enumerate}
	\item Could you mention some areas where Champernowne constant can be applied?
	\\ {\it The Champernowne constant has seemingly random numbers which are clearly well determined. This property could be useful in functional programming contexts where one would need explicit randomness which is entirely deterministic. For example, the C2 (Base 2) value of the Champernowne constant can be used as a binary random number to trigger true or false conditions randomly.}
	\item Could you mention the properties of the Champernowne constant?
	\begin{enumerate}
	{\it \item The constant given by 0.123456789101112 . . . is normal in base ten.
	\item The constant is transcendental.
	\item The constant also has a peculiar continued fraction expansion. It namely contains exceptionally large terms throughout the expansion.}
	\end{enumerate}
	\item Is Champernowne constant a Liouville number?
	\\ {\it No. Champernowne constant and Liouville numbers are both transcendental but Champernowne constant is irrational and Liouville numbers are almost rational and can be approximated quite closely by sequences of rational numbers than any algebraic irrational numbers.}
	\item Is Champernowne constant as useful as  \( \pi \) constant?
	\\ {\it The Champernowne constant is quite useful. But Pi is a constant which is there almost everywhere in mathematics. }
	\item Does it appear in any sort of geometric sense like  \( \pi \) does?
	\\ {\it The Champernowne constant cannot be represented as a finite number due to which there have been no reported utilization in geometry.}
	\item How often do you use the Champernowne constant?
	\\ {\it As a student of pure mathematics I do not use the Champernowne constant quite often.}
	\item Has the Champernowne constant ever been used in any major proofs? 
	\\ {\it No.}
	\item Can it be expressed in terms of e,  \( \pi \) or both? 
	\\ {\it No.}
	\item What works of Alan Turing and David Champernowne made use of the Champernowne constant?
	\\ {\it Notably, There were two major works of Alan Turing and David Champernowne, the TuroChamp and Round the house Chess. But the Champernowne constant is not mentioned to have been in any of these machines.}
	\item Of all the base versions of Champernowne constants which particular Champernowne sequence has been vastly used and why?
	\\{\it Base 2 (C2). Used in functional programming as a Random binary digit generator.}
	\item Who and How was Champernowne constant proved Transcendental? 
	\\ {\it The Champernowne constant was shown to be transcendental by Kurt Mahler in 1937.}
	\item What are the alternatives available for Champernowne constant?
	\\ {\it None.}
	\item Is Champernowne constant used in Turochamp or Round the house Chess? if Yes, how does it fit into the algorithms of these games?
	\\ {\it No. }
\end{enumerate}
\section*{Analysis of the Inteview}
\quad The Interviewee had moderate knowledge about the Champernowne constant. She was able to explain the basic characteristics of the constant with logical examples (mentioned in the response to the interview questions). The practical applications of the constant have been limited and the interviewee was able to explain one particular area where the constant (its base 2 value) is widely used to generate the randomized binary numbers.
\end{document}